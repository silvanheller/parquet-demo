\section{Introduction} \label{sec:introduction}

As datasets grow, so does the need for storing them efficiently and in a format that is suited to the needs of high-volume, high-velocity and high-variety data.
In this Workshop report, we will discuss Parquet\footnote{https://parquet.apache.org/} as an example of a file format suited for 'Big Data'.

\subsection{History}
Google outlined in 2010 how they were using Dremel \cite{dremel:melnik} and the Dremel file format to store high-variety structured data at petabyte scale in a columnar format.
Since Dremel is, as most Google tools are, proprietary and not open source,
Twitter and Cloudera started the Apache Parquet project as an open source implementation \cite{parquet-announcement:kestelyn}.
The goal was to provide an efficient columnar storage format for the Hadoop ecosystem.
Today, Parquet is used in production at multiple companies, the most notable ones being Stripe and Twitter \cite{adopters:parquet}.

\subsection{Using Parquet}
Since Parquet is a file format, it requires implementation and tools around it.
We have used Apache Spark \cite{spark:zaharia}, which has built-in support for Parquet and other file formats and is thus ideal to compare Parquet fairly.
The application implemented is a benchmark to compare Parquet, JSON, CSV and Apache ORC\footnote{https://orc.apache.org/}.
Our benchmark will contain two scenarios,
one with a table which has a lot of columns of which only one is accessed and the other one with nested objects.

\subsection{Outline}
Chapter \ref{sec:parquet} will describe the Parquet format in detail.
To see how Parquet compares to other data storage formats, Chapter \ref{sec:application} outlines two scenarios which will be used to benchmark Parquet.
Chapter \ref{sec:results} contains the benchmark results and Chapter \ref{sec:lessons_learned} concludes with the lessons learned.