% =======================================================================================
% =======================================================================================
% === IMPORTANT NOTE FOR STUDENTS:                                                    ===
% =======================================================================================
% === Do NOT change anything in this file. Only change files in the "reportContent"   ===
% === subfolder.                                                                      ===
% =======================================================================================
% =======================================================================================

\documentclass[runningheads]{llncs}
\usepackage{makeidx}
\usepackage{chapterbib}
%\usepackage[english,german]{babel}
%\selectlanguage{german}

\usepackage[utf8]{inputenc}
\usepackage[T1]{fontenc}
\usepackage{graphicx}\graphicspath{{images/}{reportContent/images/}{AsterixDB/images/}{CarbonData/images/}{DistributedLog/images/}{Flume/images/}{Geode/images/}{Helix/images/}{Ignite/images/}{Kudu/images/}{Kylin/images/}{Mnemonic/images/}{Omid/images/}{Parquet/images7}{Phoenix/images/}{Quickstep/images/}{S2Graph/images/}{Spark/images/}{Sqoop/images/}{Tajo/images/}{Tephra/images/}{Trafodion/images/}}

\usepackage{import}
\usepackage{lipsum}
\usepackage{setspace}

% =======================================================================================
% === The following latex-packages should cover all your needs.                       ===
% === If this is not the case, please write an email to one of the assistants in      ===
% === which you inform us which additional package you need and why.                  ===
% === (Otherwise we will not be able to generate proceedings for the course...)       ===
% =======================================================================================


% For URLs
\usepackage{url}

% For Subfigures
\usepackage{subfig}

% For Algorithms
\usepackage{algorithmic}
\usepackage{algorithm}

% For Code Snippets
\usepackage{listings}

% For Todo Notes 
\usepackage[colorinlistoftodos]{todonotes}

% For Colors (\textcolor etc.)
\usepackage{color}

% For Tables
\usepackage{multirow}

% For Tikz
\usepackage{tikz}

% For References
\usepackage{cleveref}

% Required for turning proceedings on and off
% (Source: http://tex.stackexchange.com/questions/87656/turning-parts-of-text-on-and-off)
\usepackage{etoolbox}
\usepackage{verbatim}
\newbool{produceProceedings}
\setbool{produceProceedings}{false} % Note for the assistants: Change to true in order to generate the proceedings.
\newenvironment{produceProceedings}{}{}
\ifbool{produceProceedings}{}{\AtBeginEnvironment{produceProceedings}{\comment}%
	\AtEndEnvironment{produceProceedings}{\endcomment}}

\newcommand{\coursename}{Distributed Information Systems}
\newcommand{\courseacronym}{DIS}
\newcommand{\semester}{spring semester 2017}
\newcommand{\semesterCapitals}{Spring Semester 2017}
\newcommand{\rootDocument}{15729-DIS17-Workshop\_Proceedings.tex}

\begin{document}
	
\begin{produceProceedings}
	\begin{titlepage}
		\includegraphics{UniBas_Logo_EN_Schwarz_RGB_65}
		
		\vspace{80pt}
		
		\centering
		
		\begin{spacing}{2.8}
		{\Huge \fontfamily{phv} \bfseries Proceedings of the \coursename\ (\courseacronym) Course}
		\end{spacing}
		
		\vspace{25pt}
		
		{\large Workshop Reports}
		
		\vspace{50pt}
		
		{\large \semesterCapitals}
		
		\vspace{50pt}
		
		{\large Natural Science Faculty of the University of Basel}
		
		\vspace{2pt}
		
		{\large	Department of Mathematics and Computer Science}
	\end{titlepage}	
	
%	\frontmatter 
	\pagestyle{headings}
	\addtocmark{Reports}
	%
	\chapter*{Preface}
	%
	This documents contains all student workshop reports of the \coursename\ (\courseacronym) course held at the University of Basel in the \semester.
	%
	\chapter*{Organization}
	The \coursename\ (\courseacronym) course of the \semester\ is organized by the Databases and Information Systems (DBIS) research group, Department of Mathematics and Computer Science, Univeristy
	of Basel.
	%
	\section*{Lecturer}
	Prof.\,Dr. Heiko Schuldt (heiko.schuldt@unibas.ch)
	%
	\section*{Assistants}
	M.\,Sc. Alexander Stiemer (alexander.stiemer@unibas.ch) % \\
%	M.\,Sc. Marco Vogt (marco.vogt@unibas.ch)
	%
%	\section*{Tutors}
%	M.\,Sc. \DJ or\dj e Reli\'{c} (dorde.relic@stud.unibas.ch) \\
%	M.\,Sc. Marco Vogt (marco.vogt@unibas.ch)
	%
	\tableofcontents
\end{produceProceedings}
	
%
\mainmatter
%

% Note for the assistants: Add all student reports to this list in order to generate the proceedings.
\begin{cbunit}\import{reportContent/}{reportContent.tex}\end{cbunit}

%\begin{cbunit}\import{AsterixDB/}{reportContent.tex}\end{cbunit}
%\begin{cbunit}\import{CarbonData/}{reportContent.tex}\end{cbunit}
%\begin{cbunit}\import{DistributedLog/}{reportContent.tex}\end{cbunit}
%\begin{cbunit}\import{Flume/}{reportContent.tex}\end{cbunit}
%\begin{cbunit}\import{Geode/}{reportContent.tex}\end{cbunit}
%\begin{cbunit}\import{Helix/}{reportContent.tex}\end{cbunit}
%\begin{cbunit}\import{Ignite/}{reportContent.tex}\end{cbunit}
%\begin{cbunit}\import{Kudu/}{reportContent.tex}\end{cbunit}
%\begin{cbunit}\import{Kylin/}{reportContent.tex}\end{cbunit}
%\begin{cbunit}\import{Mnemonic/}{reportContent.tex}\end{cbunit}
%\begin{cbunit}\import{Omid/}{reportContent.tex}\end{cbunit}
%\begin{cbunit}\import{Parquet/}{reportContent.tex}\end{cbunit}
%\begin{cbunit}\import{Phoenix/}{reportContent.tex}\end{cbunit}
%\begin{cbunit}\import{Quickstep/}{reportContent.tex}\end{cbunit}
%\begin{cbunit}\import{S2Graph/}{reportContent.tex}\end{cbunit}
%\begin{cbunit}\import{Spark/}{reportContent.tex}\end{cbunit}
%\begin{cbunit}\import{Sqoop/}{reportContent.tex}\end{cbunit}
%\begin{cbunit}\import{Tajo/}{reportContent.tex}\end{cbunit}
%\begin{cbunit}\import{Tephra/}{reportContent.tex}\end{cbunit}
%\begin{cbunit}\import{Trafodion/}{reportContent.tex}\end{cbunit}

\end{document}

% =======================================================================================
% =======================================================================================
% === IMPORTANT NOTE FOR STUDENTS:                                                    ===
% =======================================================================================
% === Do NOT change anything in this file. Only change files in the "reportContent"   ===
% === subfolder.                                                                      ===
% =======================================================================================
% =======================================================================================